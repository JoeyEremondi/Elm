\documentclass{article}

\usepackage[utf8]{inputenc}
\usepackage{amsthm}
\usepackage{amsmath}
\usepackage{amssymb}
\usepackage{mathtools}
%\usepackage{algorithm2e}
\usepackage[margin=1in]{geometry}
%\usepackage{algorithm2e}
%\usepackage{complexity}
%\usepackage{graphicx}
\usepackage{minted}
\usepackage{multicol} % added package


\title{APA Project 1: Monotone Frameworks\\
Forward Slicing Dead Code Elimination in Elm}
\author{Joseph Eremondi, UU\# 4229924}
\date{\today}

%\renewcommand{\thesubsection}{\thesection.\roman{subsection}}

\newtheorem{theorem}{Theorem}[section]
\newtheorem{lemma}[theorem]{Lemma}
\newtheorem{definition}[theorem]{Definition}
\newtheorem*{note}{Note}

\newcommand\abs[1]{\left|#1\right|}

\newcommand\set[1]{\left\{#1\right\}}

\newcommand\sst[0]{\mid}

\DeclarePairedDelimiter\floor{\lfloor}{\rfloor}

%\newclass{\FPT}{FPT}


\begin{document}

\maketitle

\section{Build Instructions}

To build, you must have ghc and cabal installed and on your PATH.
I built and tested with GHC 7.8.3, but I don't do anything too crazy with extensions,
so it should work with older (or newer) versions.

To compile and run my code, unzip the compressed folder, then:

\begin{minted}[%mathescape,
               %linenos,
               %numbersep=5pt,
               %gobble=2,
               %frame=lines,
               framesep=2mm]{bash}
  cd elm-compiler
  cabal sandbox init
  cabal install ./elm-compiler.cabal
  cabal install elm-make
               
\end{minted}

It's important that you run the commands in this order: the Elm make tool, which actually generates JavaScript, depends on the elm-compiler library, so you must have my modified version installed before
you install elm-make and elm-reactor from Hackage.
The dependencies are quite large, so putting them in a sandbox will allow you to remove them when you
are finished.

Note that there's quite a few dependencies for the 
Because of this, I've included Linux and Windows binaries in the elm-compiler directory. %TODO this

To build the tests, for example, SyntaxTest.elm:

\begin{minted}[%mathescape,
               %linenos,
               %numbersep=5pt,
               %gobble=2,
               %frame=lines,
               framesep=2mm]{bash}
  cd apaTests
  ../.cabal-sandbox/bin/elm-make SyntaxTest.elm --output=SyntaxTest.html --yes
               
\end{minted}

Then open SyntaxTest.html in your browser.
The first time you run a test, Elm will install its core libraries in the current directory.
The JavaScript output for a module is stored in apaTests/elm-stuff/build-artifacts/USER/PROJECT/1.0.0/ModuleName.elmo. The generated JavaScript is actually fairly readable, and shows the results of
my slicing quite easily.


\section{Control Flow}

One of the more challenging parts of this assignment was constructing the control-flow graph for a functional program.

Central to my analysis is the idea of a ``tail'' expression, that is, a sub-expression which is a possible value for a larger expression. Function parameters, If-statement guards, and the matched expression in Case statements are never tail expressions, but most others are.

For example TODO example.

Roughly speaking, an expression is treated as a statement which calculates the value of the statement, then assigns that value to whatever (possibly internal) variable is waiting for it. In my implementation, the head and tail nodes of a sub-expression are stored in a dictionary, so that when calculating the generated edges for a node, we can connect it to the control nodes of its sub-expresssions without examining those expressions directly.

Expressions or definitions which alter control flow in some way are represented as statements in the CFG. The generated nodes and edges are summarized below.

\begin{itemize}
\item
  Function Definition: A function definition has ProcEntry and ProcExit nodes. The ProcEntry flows into assigning each formal parameter to its variable value, which then flows to the function body. After the body is calculated, its result is assigned to a ``return'' parameter. Control then flows to the ProcExit node.
\item
  Function Application: Each argument is calculated in sequence, and assigned to the formal parameter value. Control then flows to the call node, which is connected to the ProcEntry of the called function. The ProcExit of that function is connected to a Return node, which assigns the return value as the value of the application as a whole.
\item
  Binary Opearations: arithmetic operations do not alter control flow. Other binary operations are treated as function applications.
\item
  If Expressions: All Elm if-statements are represented as multi-way if statements, with multiple guards and values. Each guard has a CFG edge to both the head of the next guard, and the head of its value. Tail expressions are the tails of each case body.
\item
  Case Expressions: Similar to If. The expression being matched is evaluated, and its value is assigned to an internal intermediate variable. There is then a CFG edge to each possible case. Tail expressions are the tails of each case body.
\item
  Let Expressions: Elm Let expressions are recursively scoped. Since functions are not allowed in our analysis, there can be no mutually-recursive definitions of the program halts. Let-expressions are thus sorted into a topological ordering. The right-hand-side value of each expression is evaluated, and its value assigned to each variable in the pattern on the left hand side. There are edges between the definitions (in order), with an edge from the last definition to the expression body. The value of the body is then assigned to the value of the Let-expression as a whole.
\end{itemize}

\section{The Analysis Performed}

My modified Elm compiler performs dead code removal. Monotone Frameworks are used in two ways.

\subsection{Reaching Definitions}

First, Reaching Definition Analysis is performed. The analysis follows the dataflow equations given in TODO Cite, with some small modifications:

\begin{eqnarray}
 RDef_{\circ}(l) & = & \bigsqcup \set{RDef_\bullet(l') \sst (l',l) \in CFG} 
   %\sqcup \set{\iota \sst l \in E}
        \nonumber \\
    RDef_{\bullet}(l) & = &  \hat{f_l}(RDef_\circ(l))  \nonumber \\
    RDef_{\bullet}(l_{return}) & = 
      &  \hat{f^2_{lcall, lreturn}}(RDef_\circ(lcall), RDef_\circ(lreturn))  \nonumber \\
    {f_l}(A) & = & (A \setminus kill(l)) \cup gen(l) \nonumber \\
   kill(l) & = & \set{(var, l') \sst l \text{ is an assignment to } var } \nonumber \\
   gen(l) & = & \set{(var, l) \sst l \text{ is an assignment to } var } \nonumber \\
   \hat{f_l}(\hat{A}) & = &  (f_l \circ \hat{A}) \text{ for non-procedural nodes }\nonumber \\
   \hat{f_{lcall}}(\hat{A}) & = & \lambda \delta \rightarrow
       \bigsqcup \set{\hat{A}(\delta') \sst (lcall : \delta') = \delta, \delta' \text{a valid call string of length } \leq k } \nonumber \\
   \hat{f_{lreturn}}(\hat{A_c}) & = &  TODO  \nonumber \\
\end{eqnarray}
%\begin{eqnarray}
%      
%      RDef_{\circ}(label) & = & \bigsqcup \set{RDef_\bullet(l') \sst (l',l) \in CFG} 
%        \sqcup \set{\iota \sst l \in E}  \\\\
%      RDef_{\bullet l}(A) & = &  (RDef_\circ(A) \setminus kill()) \cup gen()
%    \end{eqnarray}


First, our iota value is the empty set. This is because, in Elm, variables are never undefined. Even in imperative, monadic code, an initial value must be given when creating a reference.
Thus, there is never a point where a free variable has an undefined value. Secondly, the analysis has been made interprocedural, with the following lifted transfer functions: TODO equations Finally, after the fixed-point of the transfer function is found, we look at the reaching definitions for each expression, and remove definitions of variables not referenced in that expression.

The result of this analysis is the set, for each program point, of relevant definitions: definitions whose values reach, and are used in, an expression.

This analysis has the following properties:

\begin{itemize}
\item
 \textit{Forwards:} the property of a definition being reaching flows forwards over control-flow edges 
\item
 \textit{Context-sensitive:} call-strings of fixed length are taking into account for interprocedural analysis 
\item
 \textit{Flow-sensitive:} The ordering of the  statements is taken into account when constructing the control flow graph.
\item
 \textit{May:} the join in our lattice of results is simply set union.

\end{itemize}

\subsection{System Dependence Graph}

The actual program slicing is performed using a System Dependence Graph. Each program point has a node in the SDG. An edge $(B,C)$ will be between nodes $B$ and $C$ for one of four reasons:
\begin{itemize}
\item $C$ is data-dependent on a variable that $B$ defines. That is, $B$ was a relevant definition in the result of our first analysis for $C$. 
\item $B$ is a case or branch statement, and $C$ is a branch of $B$ (control-dependency). This is only relevant in monadic code, since functional branching dependencies are captured as data dependencies.
\item $B$ is a call node for function $f$, and $C$ is the Proc-entry node for $f$ 
\item $B$ is the Proc-exit node for function $f$, and $C$ is the return node for a call to $f$
\end{itemize} 

A definition is then relevant if there is a path from that definition to a ``target node'' in the SDG. The target nodes are the Proc-exit nodes of all functions exported by the module under analysis, as well as the main function, if one is present.

Once again, the analysis is lifted into an embellished lattice, so that a definition is relevant if there is a path to a target node which is valid with respect to function calls, for call strings of a given fixed length.

The analysis has the following properties:

\begin{itemize}
\item \textit{Backwards:} relevance as a property flows backwards along SDG edges.
\item \textit{Context-sensitive:} all-strings of fixed length are taking into account for interprocedural analysis.
\item \textit{Flow-sensitive:} The ordering of the  statements is taken into account when constructing the control flow graph, which in turn affects the dependency graph.
\item \textit{May:} the lattice join operator is set union.
\end{itemize}



\section{Limitations}

My analysis has the following limitations:

\begin{itemize}
\item
  Higher order code and closures are not supported. The analysis fails if a function is applied to fewer than its full set of arguments, or if a lambda is ever encountered in code.
\item
  Imperative (monadic) code requires that references always be variables. This basically means that the analysis supports mutable variables, but not pointers/references.
\item Monadic functions must have explicit type signatures provided.
\item
  The analysis is only applied within a module. External calls to pure functions are assumed to have no kills or gens. This is safe because Elm doesn't allow dependency cycles, meaning external calls can't refer to variables within our module. For external calls in monadic code, the kill-set is assumed to be empty, but the gen-set is assumed to contain all references passed to the function. This provides a safe over-approximation of which definitions reach given statements. TODO this
\end{itemize}

If the analysis fails at any point, the original module is returned, so the results should always be safe, if not optimal.

\section{Code Architecture}

The code is contained in the /src directory.

My changes are contained in the Optimize module, which did not exist previously in Elm. The modules contained there are:

\begin{itemize}
\item
  Optimize.Optimize: Contains the list of optimizations to be performed, and performs then in sequence. Is mainly there to allow for more optimizations to be added in the future.
\item
  Optimize.Reachability: currently unused TODO
\item
  Optimize.Types: Types common to other modules, such as Labels for Expressions, annotations for expressions, etc.
\item
  Optimize.Traversals: Functions to either transform or fold over the AST for a program, as well as some useful utilities building on those functions, such as labelling program blocks.
\item
  Optimize.Environment: Methods for traversing ASTs and adding variables to a scoped environment, so that we can deal with conflicting names properly.
\item
  Optimize.MonotoneFramework: The abstract types for flow graphs and lattices, as well as an implementation of the worklist algorithm.
\item
  Optimize.ControlFlow: Functions which take an expression and generate the control flow nodes and edges, which are then used in Reaching Definitions analysis.
\item
  Optimize.EmbellishedMonotone: Functions which take lattice operations and return lifted operations for an embellished lattice, taking into account context. Also has code to generate all valid context strings of a given length from a module.
\item
  Optimize.RelevantDefs: performs reaching definition analysis to find out, for each expression, which definitions reach that expression and are used to calculate its value.
\item
  Optimize.SDG: contains the functions which, given reaching definition information, form the System Dependence Graph for a module. Edges are followed transitively using the worklist algorithm. Functions to remove irrelevant definitions from a module are also contained here.
\end{itemize}

The only other changes to the code are minor (i.e. adding Show instances), and the addition of the optimization stage to Elm.Compiler.hs.

\section{Examples}

\subsection{Basic Syntax}

Below is an Elm module, containing the main syntactic constructs of the language.
Definitions that should be removed are named $vRel, vRel1$, etc. Those that should be removed are named
$vNotRel$ and so on.

\inputminted{elm}{apaTests/SyntaxTest.elm}

Examining the JavaScript output of the original Elm compiler, and our modified version,
we get the following diff:

Running the code, we get the following output:

%TODO output

The important detail here is that no ``variable not found" exceptions
were thrown in JavaScript, meaning that relevant definitions were not removed
from the pure code.



\inputminted{diff}{apaTests/SyntaxTest.diff}

By looking at this diff, we can see that the only lines removed are definitions of our irrelevant variables.

Another test module shows the correctness for imperative code. Here, `andThen` is Elm's version
of Haskell's monadic bind operator $>>=$.

\inputminted{elm}{apaTests/ImperativeTest.elm}          

%TODO why one not removed?
                    
\end{document}
